\documentclass[../main.tex]{subfiles}

\begin{document}
\section{Functions used in the matlab code}
\begin{tabular}{|c|c|}
    \hline
    function & This creates a function\\
    \hline
    load & Loads file\\
    \hline
    size & \\
    \hline
    sparse & creates a sparse matrix\\
    \hline
    zeros & \\
    \hline
    mod & \\
    \hline
    length & Measures length of array\\
    \hline
    abs & \\
    \hline
    sum & \\
    \hline
    max & \\
    \hline
    dot & \\
    \hline
    eye & Identity matrix\\ 
    \hline
    ~ismember & \\
    \hline
\end{tabular}

\subsection{Mesh}
Mesh parameters include three classes of values.
\begin{itemize}
    \item \textbf{Vertices:} Arrays for coordinates of mesh vertices.
    \item \textbf{Elements:} Array of cells that contain the index of vertices of element.
    \item \textbf{Boundary values:} Array of indices for the boundary nodes.(vertices)
\end{itemize}

The code first loads the mesh. \\
Then, a sparse matrix is initialised for the stiffness.\\
Next vectors for the forcing vector and solution were initialised.\\
A unity function is created.\\
\\
for loop by elements of mesh.
Variable Vert_ids and Verts for vertex identification and coordinates.\\
Calculates number of sides of vertex called n_sides.\\
Calculates area of element.\\
The centre of the of the element is computed.\\
The diameter calculated as maximum distance between two vertices.\\







\end{document}