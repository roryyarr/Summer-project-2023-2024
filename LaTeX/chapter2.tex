\documentclass{article}
\usepackage{vem-preamble}



\begin{document}
% diagramm
% E1_E2_diagram.tex
\newcommand{\MeshDiagram}[1]{%
    \begin{tikzpicture}[scale=#1]
        % Define Coordinates
        \coordinate (A) at (0.5,0);
        \coordinate (B) at (1.5,0);
        \coordinate (C) at (2,0.5*sqrt(3));
        \coordinate (D) at (1.5,sqrt(3));
        \coordinate (E) at (1.5,1+sqrt(3));
        \coordinate (F) at (0.5,1+sqrt(3));
        \coordinate (G) at (0.5,sqrt(3));
        \coordinate (H) at (0, 0.5*sqrt(3));
        
        % Axis
        \draw[help lines, color=gray!50, dashed] (0,0) grid (3,3);
        \draw[->, ultra thick] (0,0) -- (3,0) node[right] {$x$};
        \draw[->, ultra thick] (0,0) -- (0,3) node[above] {$y$};
        
        % Origin Label
        \node (O) at (0,0) [anchor=north east] {0};
        
        % Y-axis Labels
        \foreach \y in {1,2,3}
            \node (Y\y) at (0,\y) [anchor=east] {\y};
        
        % X-axis Labels
        \foreach \x in {1,2,3}
            \node (X\x) at (\x,0) [anchor=north] {\x};
        
        % Lines and Vertex Labels
        \draw (A) node[anchor=south west] {$V^{E_2}_1$} 
              -- (B) node[anchor=south east] {$V^{E_2}_2$} 
              -- (C) node[anchor=east] {$V^{E_2}_3$} 
              -- (D) node[anchor=north east] {$V^{E_2}_4$} 
              -- (E) node[anchor=north east] {$V^{E_1}_3$} 
              -- (F) node[anchor=north west] {$V^{E_1}_4$} 
              -- (G) node[anchor=north west] {$V^{E_2}_5$} 
              -- (H) node[anchor=west] {$V^{E_2}_6$} 
              -- cycle;
              
        \draw (D) node[anchor=south east] {$V^{E_1}_2$}  
              -- (G) node[anchor=south west] {$V^{E_1}_1$};
        
        % Labels for E1 and E2
        \node[black, rectangle] at (1, {0.5*sqrt(3)}) {\huge $E_2$};
        \node[blue, rectangle] at (1, {0.5 + sqrt(3)}) {\huge $E_1$};
    \end{tikzpicture}
}


\section{Example Calculation}


This section details a worked solution of the following non-homogeneous Poisson problem:
\begin{align}
    g &= 2\pi^2\sin(\pi x)\sin(\pi y)\\
    f &= \sin(\pi x)\sin(\pi y)
\end{align}



\subsection{Mesh Definition}

\begin{figure}[ht!]
    \centering
    % Call the custom command with desired scale, e.g., 3
    \MeshDiagram{2};
    \caption{Diagram showing $E_1$ and $E_2$ with labeled vertices.}
    \label{docs/LaTeX/diagram}
\end{figure}

% E1_E2_diagram.tex
\newcommand{\MeshDiagram}[1]{%
    \begin{tikzpicture}[scale=#1]
        % Define Coordinates
        \coordinate (A) at (0.5,0);
        \coordinate (B) at (1.5,0);
        \coordinate (C) at (2,0.5*sqrt(3));
        \coordinate (D) at (1.5,sqrt(3));
        \coordinate (E) at (1.5,1+sqrt(3));
        \coordinate (F) at (0.5,1+sqrt(3));
        \coordinate (G) at (0.5,sqrt(3));
        \coordinate (H) at (0, 0.5*sqrt(3));
        
        % Axis
        \draw[help lines, color=gray!50, dashed] (0,0) grid (3,3);
        \draw[->, ultra thick] (0,0) -- (3,0) node[right] {$x$};
        \draw[->, ultra thick] (0,0) -- (0,3) node[above] {$y$};
        
        % Origin Label
        \node (O) at (0,0) [anchor=north east] {0};
        
        % Y-axis Labels
        \foreach \y in {1,2,3}
            \node (Y\y) at (0,\y) [anchor=east] {\y};
        
        % X-axis Labels
        \foreach \x in {1,2,3}
            \node (X\x) at (\x,0) [anchor=north] {\x};
        
        % Lines and Vertex Labels
        \draw (A) node[anchor=south west] {$V^{E_2}_1$} 
              -- (B) node[anchor=south east] {$V^{E_2}_2$} 
              -- (C) node[anchor=east] {$V^{E_2}_3$} 
              -- (D) node[anchor=north east] {$V^{E_2}_4$} 
              -- (E) node[anchor=north east] {$V^{E_1}_3$} 
              -- (F) node[anchor=north west] {$V^{E_1}_4$} 
              -- (G) node[anchor=north west] {$V^{E_2}_5$} 
              -- (H) node[anchor=west] {$V^{E_2}_6$} 
              -- cycle;
              
        \draw (D) node[anchor=south east] {$V^{E_1}_2$}  
              -- (G) node[anchor=south west] {$V^{E_1}_1$};
        
        % Labels for E1 and E2
        \node[black, rectangle] at (1, {0.5*sqrt(3)}) {\huge $E_2$};
        \node[blue, rectangle] at (1, {0.5 + sqrt(3)}) {\huge $E_1$};
    \end{tikzpicture}
}


Consider a mesh consisting of two elements in $\mathbb{R}^2$:

\begin{itemize}
    \item \textbf{Element $E_1$}: A square with vertices labeled in counterclockwise order.
    \item \textbf{Element $E_2$}: A hexagon with vertices labeled in counterclockwise order.
\end{itemize}

The coordinates of the vertices are as follows:

$$E_1 = \begin{cases}
    V_1 = (\frac{1}{2},\sqrt{3}) \\
    V_2 = (\frac{3}{2},\sqrt{3}) \\
    V_3 = (\frac{3}{2},1+\sqrt{3}) \\ 
    V_4 = (\frac{1}{2},1+\sqrt{3}) \\
\end{cases}
\quad
E_2 = \begin{cases}
    V_1 = (\frac{1}{2},0) \\ 
    V_2 = (\frac{3}{2},0) \\ 
    V_3 = (2,\frac{\sqrt{3}}{2}) \\
    V_4 = (\frac{3}{2},\sqrt{3}) \\
    V_5 = (\frac{1}{2}, \sqrt{3}) \\ 
    V_6 = (0, \frac{\sqrt{3}}{2}) \\
\end{cases}$$\\

\subsection{Caluculations for Element E$_1$}

\subsubsection{Element Geometry}
\begin{itemize}
    \item The centroid $x_{E_1} = (1, \frac{1}{2}+\sqrt{3}).$
    \item The diameter is, $ h_{E_1} = \sqrt{2}$
    \item area is $|E_1| = 1$.
\end{itemize}

\subsubsection{Momonial basis}

\begin{enumerate}
    \item $m_1^1 = \left(\cfrac{x - 1}{\sqrt{2}}\right)^{0}\left(\cfrac{y -\frac{1}{2} - \sqrt{3}}{\sqrt{2}}\right)^{0} = 1$
    
    \item $m_x^1 = \left(\cfrac{x - 1}{\sqrt{2}}\right)^{1}\left(\cfrac{y -\frac{1}{2} - \sqrt{3}}{\sqrt{2}}\right)^{0} = \cfrac{x - 1}{\sqrt{2}}$

    \item $m_y^1 = \left(\cfrac{x - 1}{\sqrt{2}}\right)^{0}\left(\cfrac{y -(\sqrt{3} +\frac{1}{2})}{\sqrt{2}}\right)^{1} = \cfrac{y - \frac{1}{2} - \sqrt{3}}{\sqrt{2}}$
\end{enumerate}


\subsubsection{Degrees of Freedom (DOF)}

\begin{multicols}{2}
\begin{enumerate}
    \item At $E_1V_1 = \left(\frac{1}{2}, \sqrt{3}\right)$:\\
    $m_x^1(E_1V_1) = \frac{\frac{1}{2} - 1}{\sqrt{2}} = -\frac{1}{2\sqrt{2}}$\\
    $m_y^1(E_1V_1) = \frac{\sqrt{3} - \left(\sqrt{3} + \frac{1}{2}\right)}{\sqrt{2}} = -\frac{1}{2\sqrt{2}}$

    \item At $E_1V_2 = \left(\frac{3}{2}, \sqrt{3}\right)$:\\
    $m_x^1(E_1V_2) = \frac{\frac{3}{2} - 1}{\sqrt{2}} = \frac{1}{2\sqrt{2}}$ \\
    $m_y^1(E_1V_2) = \frac{\sqrt{3} - \left(\frac{1}{2} + \sqrt{3}\right)}{\sqrt{2}} = -\frac{1}{2\sqrt{2}}$

    \item At $E_1V_3 = \left(\frac{3}{2}, 1 + \sqrt{3}\right)$:\\
    $m_x^1(E_1V_3) = \frac{\frac{3}{2} - 1}{\sqrt{2}} = \frac{1}{2\sqrt{2}}$ \\
    $m_y^1(E_1V_3) = \frac{1 + \sqrt{3} - \left(\sqrt{3} + \frac{1}{2}\right)}{\sqrt{2}} = \frac{1}{2\sqrt{2}}$

    \item At $E_1V_4 = \left(\frac{1}{2}, 1 + \sqrt{3}\right)$:\\
    $m_x^1(E_1V_4) = \frac{\frac{1}{2} - 1}{\sqrt{2}} = -\frac{1}{2\sqrt{2}}$ \\
    $m_y^1(E_1V_4) = \frac{1 + \sqrt{3} - \left(\sqrt{3} + \frac{1}{2}\right)}{\sqrt{2}} = \frac{1}{2\sqrt{2}}$
\end{enumerate}
\end{multicols}


\subsubsection{D$_1$ Matrix}
$$D_1 = \begin{pmatrix}
    m_1^1(V_1) & m_x^1(V_1) & m_y^1(V_1)\\
    m_1^1(V_2) & m_x^1(V_2) & m_y^1(V_2)\\
    m_1^1(V_3) & m_x^1(V_3) & m_y^1(V_3)\\
    m_1^1(V_4) & m_x^1(V_4) & m_y^1(V_4)\\
\end{pmatrix} \quad = \quad
\begin{pmatrix}
    1 & -\frac{1}{2\sqrt{2}} & -\frac{1}{2\sqrt{2}}\\
    1 & \frac{1}{2\sqrt{2}} & -\frac{1}{2\sqrt{2}}\\
    1 & \frac{1}{2\sqrt{2}} & \frac{1}{2\sqrt{2}}\\
    1 & -\frac{1}{2\sqrt{2}} & \frac{1}{2\sqrt{2}}\\
\end{pmatrix}$$


\subsubsection{Gradient of Monomial Basis}
\begin{enumerate}
    \item $\nabla m_1^1 = (0,0)$
    \item $\nabla m_x^1 = (\frac{1}{\sqrt{2}},0)$ 
    \item $\nabla m_y^1 = (0,\frac{1}{\sqrt{2}})$ 
\end{enumerate}


\subsubsection{Local Basis Functions $\boldsymbol{\varphi}$}
Given $\varphi_i(V_j^{E_1}) = \delta_{ij}, \quad \forall i,j = 1, \ldots ,4. \quad \varphi$  construct each basis function at each vertex using the follwoing equation:
$$\varphi_i = \cfrac{(x_{i+1}-x)(y_{i+1}-y)}{(x_{i+1}-x_{i})(y_{i+1}-y_{i})}$$\\
where $(x_i, y_i)$ and $(x_{i+1}, y_{i+1})$ are the coordinate for the points $V_i$ and $V_{i+1}^{E_1}$ respectively. Moreover  
Because $E_1$ is square with side lengths of 1 the denominator $(x_{i+1}-x_{i})(y_{i+1}-y_{i}) = 1, \quad \forall i.$ Hence the basis functions are calculated as follows:
\begin{multicols}{2}
    \begin{enumerate}
        \item $\varphi_1 = (\frac{3}{2} - x)(1 + \sqrt{3} - y)$
        \item $\varphi_2 = (x - \frac{1}{2})(1 + \sqrt{3} - y)$
        \item $\varphi_3 = (x - \frac{1}{2} )(y - \sqrt{3})$
        \item $\varphi_4 = (\frac{3}{2} - x)(y - \sqrt{3})$
    \end{enumerate}
\end{multicols}


\subsubsection{Gradient of E$_1$ Local Basis Functions}
\begin{multicols}{2}
    \begin{enumerate}
        \item $\nabla\varphi_1 = \left(-(1 + \sqrt{3} - y),  -(\frac{3}{2}-x)\right)$
        \item $\nabla\varphi_2 = \left(1 + \sqrt{3} - y,-(x - \frac{1}{2})\right)$
        \item $\nabla\varphi_3 = \left(y - \sqrt{3},x - \frac{1}{2}\right)$
        \item $\nabla\varphi_4 = \left(-(y-\sqrt{3}),\frac{3}{2}-x\right)$
    \end{enumerate}
\end{multicols}


\subsubsection{Projection $\boldsymbol{P_0\varphi_i}$}
The projection is defined as follows:
$$P_0\varphi_j = \cfrac{1}{4}\sum_{j=1}^4\varphi_i(V_j^{E_1})$$
Subsequently, $P_0\varphi_i = \frac{1}{4}, \quad \forall i$ because  $\varphi_i(V_j^{E_1}) = \delta_{ij}$


\subsubsection{Inner Products $(\nabla m_i^1, \nabla\varphi_j)_{E_1}$}
\begin{enumerate}
    \item \begin{align*}
        (\nabla_x^1,\nabla\varphi_1)_{E_1} =& \iint_{E_1}\nabla m_x^1 \cdot \nabla\varphi_1 \,dydx \\
        & \int_{\frac{1}{2}}^\frac{3}{2}\int_{\sqrt{3}}^{\sqrt{3}+1} \left(\frac{1}{2},0\right)\cdot \left(-\left(1 + \sqrt{3} - y\right),  -\left(\frac{3}{2}-x\right)\right) \,dydy \\
    \end{align*}

    \item \begin{align*}
        (\nabla m_x^1,\nabla\varphi_2)_{E_1} =& \iint_{E_1}\nabla m_x^1 \cdot \nabla\varphi_2 \,dydx \\
        & \int_{\frac{1}{2}}^\frac{3}{2}\int_{\sqrt{3}}^{\sqrt{3}+1} \left(\frac{1}{2},0\right)\cdot \left(-\left(1 + \sqrt{3} - y\right),  -\left(\frac{3}{2}-x\right)\right) \,dydy \\
    \end{align*}

    \item \begin{align*}
        (\nabla m_x^1,\nabla\varphi_2)_{E_1} =& \iint_{E_1}\nabla m_x^1 \cdot \nabla\varphi_2 \,dydx \\
        & \int_{\frac{1}{2}}^\frac{3}{2}\int_{\sqrt{3}}^{\sqrt{3}+1} \left(\frac{1}{2},0\right)\cdot \left(-\left(1 + \sqrt{3} - y\right),  -\left(\frac{3}{2}-x\right)\right) \,dydy \\
    \end{align*}

    \item \begin{align*}
        (\nabla m_x^1,\nabla\varphi_4)_{E_1} =& \iint_{E_1}\nabla m_x^1 \cdot \nabla\varphi_4 \,dydx \\
        & \int_{\frac{1}{2}}^\frac{3}{2}\int_{\sqrt{3}}^{\sqrt{3}+1} \left(\frac{1}{2},0\right)\cdot \left(-\left(1 + \sqrt{3} - y\right),  -\left(\frac{3}{2}-x\right)\right) \,dydy \\
    \end{align*}
\end{enumerate}


\subsubsection{$(\nabla m_x,\nabla\phi)$}
$(\nabla m_x^1,\nabla\phi_1)_{E_1} = 
\left(\frac{1}{\sqrt{2}},0\right)\cdot\left(-(1 + \sqrt{3} - y),  -(\frac{3}{2}-x)\right) = 
-\frac{1}{\sqrt{2}}(1 + \sqrt{3} - y)|_{y=\sqrt{3}} = 
-\frac{1}{\sqrt{2}}$\\
$(\nabla m_x^1,\nabla\phi_2)_{E_1} = 
\left(\frac{1}{\sqrt{2}},0\right)\cdot\left(1 + \sqrt{3} - y),  -(x-\frac{1}{2})\right) = 
\frac{1}{\sqrt{2}}(1 + \sqrt{3} - y)|_{y=\sqrt{3}} = 
\frac{1}{\sqrt{2}}$\\
$(\nabla m_x^1,\nabla\phi_3)_{E_1} = 
\left(\frac{1}{\sqrt{2}},0\right)\cdot\left(y-\sqrt{3}, x-\frac{1}{2}\right) = 
\frac{1}{\sqrt{2}}(y-\sqrt{3})|_{y=\sqrt{3}+1} = 
-\frac{1}{\sqrt{2}}$\\
$(\nabla m_x^1,\nabla\phi_3)_{E_1} = 
\left(\frac{1}{\sqrt{2}},0\right)\cdot\left(-(y-\sqrt{3}), \frac{3}{2}-x)\right) = 
-\frac{1}{\sqrt{2}}(y-\sqrt{3})|_{y=\sqrt{3}+1} = 
\frac{1}{\sqrt{2}}$\\


\subsubsection{$\mathbf{(\nabla m_y,\nabla\phi)}$}
$(\nabla m_y^1,\nabla\phi_1)_{E_1} = 
\left(0,\frac{1}{\sqrt{2}}\right)\cdot\left(-(1 + \sqrt{3} - y),  -(\frac{3}{2}-x)\right) = 
-\frac{1}{\sqrt{2}}(\frac{3}{2}-x)|_{x=\frac{1}{2}} = 
-\frac{1}{\sqrt{2}}$\\
$(\nabla m_y^1,\nabla\phi_2)_{E_1} = 
\left(0,\frac{1}{\sqrt{2}}\right)\cdot\left(1 + \sqrt{3} - y),  -(x-\frac{1}{2})\right) = 
-\frac{1}{\sqrt{2}}(x - \frac{1}{2})|_{x=\frac{3}{2}} = 
-\frac{1}{\sqrt{2}}$\\
$(\nabla m_y^1,\nabla\phi_3)_{E_1} = 
\left(0,\frac{1}{\sqrt{2}}\right)\cdot\left(y-\sqrt{3}, x-\frac{1}{2}\right) = 
\frac{1}{\sqrt{2}}(x-\frac{1}{2})|_{x=\frac{3}{2}} = 
\frac{1}{\sqrt{2}}$\\
$(\nabla m_y^1,\nabla\phi_3)_{E_1} = 
\left(0,\frac{1}{\sqrt{2}}\right)\cdot\left(-(y-\sqrt{3}), \frac{3}{2}-x)\right) = 
\frac{1}{\sqrt{2}}(\frac{3}{2}-x)|_{x=\frac{1}{2}} = 
\frac{1}{\sqrt{2}}$\\

\subsubsection{B$_1$ Matrix}
Next the $B_1$ matrix can be constructed.\\
$B_1 = \begin{pmatrix}
    P_0\varphi_1 & P_0\varphi_2 & P_0\varphi_3 & P_0\varphi_4\\
    (\nabla m_x^1,\nabla\phi_1)_{E_1} & (\nabla m_x^1,\nabla\phi_2)_{E_1} & (\nabla m_x^1,\nabla\phi_3)_{E_1} & (\nabla m_x^1,\nabla\phi_4)_{E_1}\\
    (\nabla m_y^1,\nabla\phi_1)_{E_1} & (\nabla m_y^1,\nabla\phi_2)_{E_1} & (\nabla m_y^1,\nabla\phi_3)_{E_1} & (\nabla m_y^1,\nabla\phi_4)_{E_1}\\
\end{pmatrix} =
\frac{1}{4}\begin{pmatrix}
    1 & 1 & 1 & 1 \\
    -\frac{\sqrt{2}}{2} & \frac{\sqrt{2}}{2} & -\frac{\sqrt{2}}{2} & \frac{\sqrt{2}}{2}\\
    -\frac{\sqrt{2}}{2} & -\frac{\sqrt{2}}{2} & \frac{\sqrt{2}}{2} & \frac{\sqrt{2}}{2}
\end{pmatrix}$

\subsubsection{G Matrix}
\begin{align*}
    G_1 =& BD =\cfrac{1}{\sqrt{2}}\frac{1}{4}
    \begin{pmatrix}
        1 & 1 & 1 & 1 \\
        -\frac{\sqrt{2}}{2} & \frac{\sqrt{2}}{2} & -\frac{\sqrt{2}}{2} & \frac{\sqrt{2}}{2}\\
        -\frac{\sqrt{2}}{2} & -\frac{\sqrt{2}}{2} & \frac{\sqrt{2}}{2} & \frac{\sqrt{2}}{2}
    \end{pmatrix}\begin{pmatrix}
        \sqrt{2} & -1 & -1\\
        \sqrt{2} & 1 & -1\\
        \sqrt{2} & 1 & 1\\
        \sqrt{2} & -1 & 1\\
    \end{pmatrix}\\
    =& \begin{pmatrix}
        1 & 0 & 1 \\
        0 & 1 & 1 \\
        0 & 0 & 1 \\
    \end{pmatrix}
\end{align*}

\subsection{Calculations for the Element E$_2$.}

\subsubsection{Element Geometry}
\begin{itemize}
    \item The centriod $x_{E_2} = (\frac{1}{2}, \frac{1}{2}).$
    \item The diameter $ h_{E_2} = \sqrt{2}$
    \item The area $|E_2| = 1$.
\end{itemize}


\subsubsection{Monomial Basis Functions}
$
\noindent\hspace*{25pt} m_1^2 = 1\\
\hspace*{25pt} m_x^2 = \cfrac{x-1}{2}\\
\hspace*{25pt} m_y^2 = \cfrac{y-1}{2}
$

\subsubsection{DOFs for E$_2$}
$\begin{array}{ll}
m_x^2|_{x=0}\hspace{1.5pt} = \cfrac{0-1}{2} = -\frac{1}{2} &\qquad m_y^2|_{y=0} \hspace{7pt}= \cfrac{0-1}{2} = -\frac{1}{2} \\
m_x^2|_{x=\frac{1}{2}} = \cfrac{\frac{1}{2}-1}{2} = -\frac{1}{4} &\qquad m_y^2|_{y=\frac{\sqrt{3}}{2}} = \cfrac{\frac{\sqrt{3}}{2}-1}{2} = \frac{\sqrt{3}-2}{4} \\
m_x^2|_{x=\frac{3}{2}} = \cfrac{\frac{3}{2}-1}{2} = \frac{1}{4} &\qquad m_y^2|_{y=\sqrt{3}}\, = \cfrac{\sqrt{3}-1}{2} = \frac{\sqrt{3}-1}{2} \\
m_x^2|_{x=2}\hspace{1.5pt} = \cfrac{2-1}{2} = \frac{1}{2} &
\end{array}$

\subsubsection{D Matrix}

$D_2 = \cfrac{1}{4}\begin{pmatrix}
    4 & -1 & - 2\\
    4 & 1 & - 2\\
    4 & 2 & \sqrt{3} - 2\\
    4 & 1 & 2\sqrt{3} - 2\\
    4 & -1 & 2\sqrt{2} - 2\\
    4 & -2 & \sqrt{3} - 2\\
\end{pmatrix}$

\subsection{B matrix}
$\nabla m_1^2 = (0,0)$\\
$\nabla m_x^2 = (\frac{1}{2},0)$ \\
$\nabla m_y^2 = (0,\frac{1}{2})$ \\

Hence, \\
$P_0m_1 = \int_E\frac{1}{E}$



% \subsubsection{}

\section{Exact Solution}

Suppose the solution is the form of $u(x,y) = A\sin(\pi x)\sin(\pi y)$\\
\begin{enumerate}
    \item $\Delta u(x,y) = \cfrac{\partial^2 u}{\partial x^2} + \cfrac{\partial^2 u}{\partial y^2}$
    \item where: $\cfrac{\partial^2 u}{\partial x^2} = -A\pi^2\sin(\pi x)\sin(\pi y)$
    \item and: $\cfrac{\partial^2 u}{\partial x^2} = -A\pi^2\sin(\pi x)\sin(\pi y)$
    Hence $\Delta u(x,y) = -2A\pi^2\sin(\pi x)\sin(\pi y)$
    \item Substituting the Poisson equation: $2A\pi^2\sin(\pi x)\sin(\pi y) = 2\pi^2\sin(\pi x)\sin(\pi y)$
    \item Hence A = 1
\end{enumerate}

\subsection{Solution at vertices}

\begin{enumerate}
    \item $u(1/2,0) = 0$
    \item $(3/2,0) = 0$
    \item $(2,\sqrt{3}/2) = 0$
    \item $(3/2,\sqrt{3}) = \sin(3/2\pi)\sin(\sqrt{3}\pi) = 0.746$
    \item $(3/2,1+\sqrt{3}) = \sin({3\pi/2})\sin(1+\sqrt{3}) = -0.398$
    \item $(1/2,1+\sqrt{3}) = \sin(\pi/2)\sin(1+\sqrt{3}) = 0.398$
    \item $(1/2,\sqrt{3}) = \sin(\pi/2)\sin(1+\sqrt{3}) = -0.746$
    \item $(0, \sqrt{3}/2) = 0$
\end{enumerate}

\end{document}