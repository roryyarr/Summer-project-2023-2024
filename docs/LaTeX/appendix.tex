\chapter{Appendix}

% Reset section counter and modify numbering for the appendix
\setcounter{section}{0}                 % Reset section counter to 0
\renewcommand{\thesection}{\Alph{section}} % Section numbering as A, B, C, etc.
\renewcommand{\thesubsection}{\Alph{section}\arabic{subsection}} % Subsections as A1, A2, etc.
\renewcommand{\thesubsubsection}{\Alph{section}\arabic{subsection}.\arabic{subsubsection}} % Subsubsections as A1.1, A1.2, etc.


% Appendix A
\section{First Appendix}
This section shows the MATLAB code for the VEM method.

\subsection{The main VEM method}

\begin{lstlisting}[style=MatlabStyle]
function u = vem(mesh_filepath, rhs, boundary_condition)
% VEM computes the virtual element solution of a Poisson problem on a polygonal mesh

% SYNOPSIS: u = vem(mesh_filepath, rhs, boundary_condition)
%
% INPUT: mesh_filepath: A string specifying the path to a mesh file
%        rhs:           A handle to a function specifying the PDE forcing function
%        boundary_condition: A handle to a function specifying the boundary condition of the PDE
% 
% OUTPUT: u: A vector of the the degrees of freedom of the virtual element solution to the PDE

mesh = load(mesh_filepath); % Load the mesh from a .mat file
n_dofs = size(mesh.vertices, 1); n_polys = 3; % Method has 1 degree of freedom per vertex
K = sparse(n_dofs, n_dofs); % Stiffness matrix
F = zeros(n_dofs, 1); % Forcing vector
u = zeros(n_dofs, 1); % Degrees of freedom of the virtual element solution
linear_polynomials = {[0,0], [1,0], [0,1]}; % Impose an ordering on the linear polynomials
mod_wrap = @(x, a) mod(x-1, a) + 1; % Utility function for wrapping around a vector

% Loop through elements to compute local stiffness matrix and forcing vector
for el_id = 1:length(mesh.elements)
    vert_ids = mesh.elements{el_id}; % Global IDs of the vertices of this element
    verts = mesh.vertices(vert_ids, :); % Coordinates of the vertices of this element
    n_sides = length(vert_ids); % Start computing the geometric information
    area_components = verts(:,1) .* verts([2:end,1],2) - verts([2:end,1],1) .* verts(:,2);
    area = 0.5 * abs(sum(area_components));
    centroid = sum((verts + verts([2:end,1],:)) .* repmat(area_components,1,2)) / (6*area);
    diameter = 0; % Compute the diameter by looking at every pair of vertices
    for i = 1:(n_sides-1)
        for j = (i+1):n_sides
            diameter = max(diameter, norm(verts(i, :) - verts(j, :)));
        end
    end

    % Initialize D and B matrices for projection
    D = zeros(n_sides, n_polys); D(:, 1) = 1;
    B = zeros(n_polys, n_sides); B(1, :) = 1/n_sides;
    
    % Further calculations for polynomials, gradients, and projection
    for vertex_id = 1:n_sides
        vert = verts(vertex_id, :); % Current vertex and its neighbors
        prev = verts(mod_wrap(vertex_id - 1, n_sides), :);
        next = verts(mod_wrap(vertex_id + 1, n_sides), :);
        vertex_normal = [next(2) - prev(2), prev(1) - next(1)]; % Edge normal
        for poly_id = 2:n_polys
            poly_degree = linear_polynomials{poly_id};
            monomial_grad = poly_degree / diameter; % Gradient of linear polynomial
            D(vertex_id, poly_id) = dot(vert - centroid, poly_degree) / diameter;
            B(poly_id, vertex_id) = 0.5 * dot(monomial_grad, vertex_normal);
        end
    end

    % Assemble the global stiffness matrix and forcing vector
    projector = (B*D) \ B; % Compute the local Ritz projector to polynomials
    stabilising_term = (eye(n_sides) - D * projector)' * (eye(n_sides) - D * projector);
    G = B*D; G(1, :) = 0;
    local_stiffness = projector' * G * projector + stabilising_term;
    K(vert_ids,vert_ids) = K(vert_ids,vert_ids) + local_stiffness; % Copy local to global
    F(vert_ids) = F(vert_ids) + rhs(centroid) * area / n_sides;
end

% Apply boundary conditions and solve the linear system
boundary_vals = boundary_condition(mesh.vertices(mesh.boundary, :));
internal_dofs = ~ismember(1:n_dofs, mesh.boundary);
F = F - K(:, mesh.boundary) * boundary_vals; % Apply the boundary condition
u(internal_dofs) = K(internal_dofs, internal_dofs) \ F(internal_dofs); % Solve
u(mesh.boundary) = boundary_vals; % Set the boundary values
plot_solution(mesh, u)
end
\end{lstlisting}

% Appendix B
\section{Second Appendix}
This section shows the Julia code used in this report. 

\subsection{Mesh Struct}
This section shows the definition of the Mesh structure.


\begin{jllisting}[style=JuliaStyle]
struct Mesh
    boundary::Vector{Int64}    # Indices of boundary vertices
    elements::Vector{Vector{Int64}}  # Indices of vertices for each element
    vertices::Vector{Tuple{Float64, Float64}}  # Coordinates of vertices
end
\end{jllisting}

\subsection{Main VEM method}
This is the main Virtual Element Method. It implements the theoretical technique discussed in the report.

\begin{jllisting}[style=JuliaStyle]
function vem(mesh::Mesh, rhs::Function, boundary_condition::Function; debug::Bool=false, debug_file_path::String="vem_debug_output.md")::Vector{Float64}
    # Initialize problem dimensions
    n_dofs = length(mesh.vertices)  # Number of degrees of freedom, one per vertex
    n_polys = 3  # Number of polynomials in the VEM space (constant + linear terms)

    # Initialize global stiffness matrix, forcing vector, and solution vector
    K = spzeros(Float64, n_dofs, n_dofs)  # Global stiffness matrix
    F = zeros(n_dofs)  # Global forcing vector
    u = zeros(n_dofs)  # Solution vector (degrees of freedom)

    # Linear polynomials (constant, x, y)
    linear_polynomials = [[0, 0], [1, 0], [0, 1]]  # Polynomial basis functions

    # Debugging: open a debug file if enabled
    if debug
        debug_file = open(debug_file_path, "w")
    end

    # Write the table of contents (TOC) for the debug file if debugging is enabled
    if debug
        write(debug_file, "# Debug Output for VEM Method\n\n")
        write(debug_file, "## Table of Contents\n\n")
        write(debug_file, "- [Number of DOFs](#number-of-dofs)\n")
        write(debug_file, "- [Initial Stiffness Matrix (K)](#initial-stiffness-matrix)\n")
        write(debug_file, "- [Initial Forcing Vector (F)](#initial-forcing-vector)\n")

        # Generate TOC dynamically for each element in the mesh
        for el_id in 1:length(mesh.elements)
            write(debug_file, "- [Element $el_id Details](#element-$el_id-details)\n")
            write(debug_file, "  - [Area, Centroid, Diameter](#element-$el_id-area-centroid-diameter)\n")
            write(debug_file, "  - [D and B Matrices](#element-$el_id-d-and-b-matrices)\n")
            write(debug_file, "  - [K and F Matrices](#element-$el_id-k-and-f-matrices)\n")
        end
        write(debug_file, "- [Boundary Conditions](#boundary-conditions)\n")
        write(debug_file, "- [Final Solution (u)](#final-solution)\n\n")
    end

    # Debugging: initial values of K, F, and u
    if debug
        write(debug_file, "[Back to top](#table-of-contents)\n\n")
        write(debug_file, "## Number of DOFs\nNumber of DOFs: $n_dofs\n\n")
        write(debug_file, "## Initial Stiffness Matrix\nInitial K: $K\n\n")
        write(debug_file, "## Initial Forcing Vector\nInitial F: $F\n\n")
        write(debug_file, "## Initial Degrees of Freedom (u)\nInitial u: $u\n\n")
    end

    # Loop over all elements in the mesh to compute local contributions to K and F
    for el_id in 1:length(mesh.elements)
        vert_ids = mesh.elements[el_id]  # Global vertex IDs of this element
        verts = [mesh.vertices[v] for v in vert_ids]  # Coordinates of the element's vertices
        n_sides = length(vert_ids)  # Number of sides of the polygon (element)

        # Debugging: print element details
        if debug
            write(debug_file, "[Back to top](#table-of-contents)\n\n")
            write(debug_file, "## Element $el_id Details\n")
            write(debug_file, "- Vertex IDs: $vert_ids\n")
            write(debug_file, "- Vertices: $verts\n")
            write(debug_file, "- Number of sides: $n_sides\n\n")
        end

        # Compute geometric properties of the element (area, centroid, and diameter)
        verts_array = hcat([collect(v) for v in verts]...)'  # Convert vertex coordinates to matrix form
        area_components = verts_array[:,1] .* verts_array[[2:end; 1],2] .- verts_array[[2:end; 1],1] .* verts_array[:,2]
        area = 0.5 * abs(sum(area_components))  # Compute the area of the polygon

        # Debugging: print geometric details
        if debug
            write(debug_file, "[Back to top](#table-of-contents)\n\n")
            write(debug_file, "### Element $el_id Area, Centroid, and Diameter\n")
            write(debug_file, "- Area components: $area_components\n")
            write(debug_file, "- Area: $area\n")
        end

        # Compute the centroid of the polygon
        centroid = sum((verts_array .+ verts_array[[2:end; 1],:]) .* repeat(area_components, 1, 2), dims=1) / (6 * area)

        # Compute the diameter of the polygon (largest distance between two vertices)
        diameter = 0.0
        for i in 1:(n_sides-1)
            for j in (i+1):n_sides
                diameter = max(diameter, norm(verts_array[i, :] - verts_array[j, :]))
            end
        end

        # Debugging: print centroid and diameter details
        if debug
            write(debug_file, "- Centroid: $centroid\n")
            write(debug_file, "- Diameter: $diameter\n\n")
        end

        # Initialize D and B matrices for the element
        D = zeros(n_sides, n_polys)  # Matrix D for polynomial projections
        D[:, 1] .= 1  # Constant term in polynomial
        B = zeros(n_polys, n_sides)  # Matrix B for polynomial evaluation
        B[1, :] .= 1 / n_sides  # Constant term for all sides

        # Debugging: print initial D and B matrices
        if debug
            write(debug_file, "[Back to top](#table-of-contents)\n\n")
            write(debug_file, "### Element $el_id D and B Matrices\n")
            write(debug_file, "- D Matrix: $D\n")
            write(debug_file, "- B Matrix: $B\n")
        end

        # Fill the D and B matrices with values for each vertex of the element
        for vertex_id in 1:n_sides
            vert = verts[vertex_id]
            prev = verts[mod(vertex_id - 2, n_sides) + 1]  # Previous vertex
            next = verts[mod(vertex_id, n_sides) + 1]  # Next vertex

            # Coordinates of current, previous, and next vertices
            vert_x, vert_y = vert
            prev_x, prev_y = prev
            next_x, next_y = next

            vertex_normal = [next_y - prev_y, prev_x - next_x]  # Normal vector to the edge
            centroid_vec = vec(centroid)  # Centroid vector

            # Fill D and B for higher-degree polynomials (linear terms)
            for poly_id in 2:n_polys
                poly_degree = linear_polynomials[poly_id]  # Polynomial degrees
                monomial_grad = poly_degree / diameter  # Gradient of the monomial

                D[vertex_id, poly_id] = dot([vert_x, vert_y] .- centroid_vec, poly_degree) / diameter
                B[poly_id, vertex_id] = 0.5 * dot(monomial_grad, vertex_normal)
            end
        end

        # Compute the projector and stabilizing term for the stiffness matrix
        projector = (B * D) \ B  # Projector matrix
        stabilising_term = (I - D * projector)' * (I - D * projector)  # Stabilizing term for the stiffness matrix

        # G matrix for additional stability
        G = B * D
        G[1, :] .= 0  # Set the first row to 0 (to satisfy certain stability conditions)

        # Compute the local stiffness matrix for this element
        local_stiffness = projector' * G * projector + stabilising_term

        # Debugging: print projector, G matrix, and local stiffness matrix
        if debug
            write(debug_file, "[Back to top](#table-of-contents)\n\n")
            write(debug_file, "- Projector: $projector\n")
            write(debug_file, "- Stabilising Term: $stabilising_term\n")
            write(debug_file, "- G Matrix: $G\n")
            write(debug_file, "- Local Stiffness: $local_stiffness\n\n")
        end

        # Assemble the local stiffness matrix into the global matrix K
        for i in 1:length(vert_ids)
            for j in 1:length(vert_ids)
                K[vert_ids[i], vert_ids[j]] += local_stiffness[i, j]
            end
        end

        # Compute the contribution to the global forcing vector F
        rhs_value = rhs(centroid)  # Evaluate the right-hand side function at the centroid
        F[vert_ids] .+= rhs_value * area / n_sides

        # Debugging: print element contribution to K and F
        if debug
            write(debug_file, "[Back to top](#table-of-contents)\n\n")
            write(debug_file, "### Element $el_id K and F Matrices\n")
            write(debug_file, "- RHS value: $rhs_value\n")
            write(debug_file, "- Updated K and F Matrices\n\n")
        end
    end

    # Apply boundary conditions
    boundary_values = boundary_condition(mesh.vertices)
    if debug
        write(debug_file, "[Back to top](#table-of-contents)\n\n")
        write(debug_file, "## Boundary Conditions\n")
        write(debug_file, "- Boundary Values: $boundary_values\n\n")
    end

    for idx in mesh.boundary
        u[idx] = boundary_values[idx]  # Set solution values at boundary
        K[idx, :] .= 0  # Zero out the row in K
        K[idx, idx] = 1.0  # Set the diagonal entry to 1
        F[idx] = boundary_values[idx]  # Set the corresponding entry in F
    end

    # Solve the system K * u = F to find the solution
    u = K \ F

    # Debugging: print the final solution
    if debug
        write(debug_file, "[Back to top](#table-of-contents)\n\n")
        write(debug_file, "## Final Solution\n")
        write(debug_file, "- Final Solution u: $u\n")
    end

    # Close the debug file if it was opened
    if debug
        close(debug_file)
    end

    return u  # Return the computed solution
end
\end{jllisting}

\subsection{Plot Functions}
Here is the plotting functions used. 

\subsubsection{Heatmap Plot}
Unlike Matlab Julia does not have a equivalent function to patches. Which complicated the plotting function. 

\begin{jllisting}[style=JuliaStyle]
function plot_heatmap(mesh::Mesh, solution::Vector{Float64}; ptitle =nothing, axis_labels = nothing, colourscheme=:blues, show_colourbar=true, savepath=nothing)
    # Extract the vertices and solution values
    x_coords = [v[1] for v in mesh.vertices]
    y_coords = [v[2] for v in mesh.vertices]

    # Calculate the axis limits based on the mesh vertices
    x_min, x_max = minimum(x_coords), maximum(x_coords)
    y_min, y_max = minimum(y_coords), maximum(y_coords)

    # Create a plot object with aspect ratio and axis limits, conditionally showing the color bar
    p = plot(legend = false, aspect_ratio = :equal, xlims=(x_min, x_max), ylims=(y_min, y_max), 
             colorbar = show_colourbar ? :right : false)

    # Loop through each element to fill with interpolated color
    for element in mesh.elements
        # Extract the x, y coordinates of the vertices in the element
        x_elem = x_coords[element]
        y_elem = y_coords[element]
        
        # Get the solution values for the vertices of the element
        solution_elem = solution[element]

        # Fill the polygon (triangle/polygon in this case) with interpolated color
        plot!(p, x_elem, y_elem, fill_z = solution_elem, seriestype = :shape, lw = 1, colour=colourscheme)
    end

    # Final plot customization
    if axis_labels !== nothing
        xlabel!(p, axis_labels[1])
        ylabel!(p, axis_labels[2])
    end
    if ptitle !== nothing
        title!(p, ptitle)
    end


    # Save the plot if savepath is provided, otherwise display it
    if savepath !== nothing
        savefig(p, savepath)
    else
        display(p)
    end
end
\end{jllisting}

\subsubsection{Wireframe Plot}
This is the wireframe plot.

\begin{jllisting}[style=JuliaStyle]
function plot_wireframe(mesh::Mesh, heights::Vector{Float64}; ptitle=nothing, azimuth=30, elevation=30, solidcolour=nothing, colourscheme=:viridis, savepath=nothing)
    # Extract vertices coordinates
    x_coords = [v[1] for v in mesh.vertices]
    y_coords = [v[2] for v in mesh.vertices]

    # Normalize heights for color mapping
    min_h = minimum(heights)
    max_h = maximum(heights)
    norm_heights = (heights .- min_h) ./ (max_h - min_h)

    # Initialize 3D plot with camera position
    if ptitle !== nothing
        fig = plot3d(title=ptitle, legend=false, camera=(azimuth, elevation))
    else
        fig = plot3d(legend=false, camera=(azimuth, elevation))
    end
    # Loop through each element and plot the wireframe
    for element in mesh.elements
        # Get the x, y, and z coordinates for the current element
        x_el = [x_coords[i] for i in element]
        y_el = [y_coords[i] for i in element]
        z_el = [heights[i] for i in element]

        # Close the polygon by adding the first vertex at the end
        append!(x_el, x_el[1])
        append!(y_el, y_el[1])
        append!(z_el, z_el[1])

        # Determine color based on provided inputs
        if solidcolour !== nothing
            # Use the solid color if provided
            line_colour = solidcolour
        elseif colourscheme !== nothing
            # Use the color scheme if no solid color is provided
            avg_height = mean(norm_heights[element])  # Average height for the element
            line_colour = cgrad(colourscheme)[round(Int, avg_height * 255) + 1]  # Color from gradient
        else
            error("Please provide either a valid color scheme or a solid color.")
        end

        # Plot the edges of the polygon
        plot3d!(x_el, y_el, z_el, linecolor=line_colour, linewidth=1.5)
    end

    # Save the plot if savepath is provided, otherwise display it
    if savepath !== nothing
        savefig(fig, savepath)
    else
        display(fig)
    end
end
\end{jllisting}

