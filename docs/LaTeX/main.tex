\documentclass{article}
\usepackage{vem-preamble}

\title{VEM summer project}
\author{Rory Yarr}
\date{November 2023}


\begin{document}

\maketitle

\tableofcontents

\section{Introduction}

\subsection{Poisson problem}
\begin{align*}
    -\Delta{u} & = f \quad  \text{in}\ \ \Omega \\
    u & = 0 \quad  \text{on}\ \delta\Omega
\end{align*}

Where $\Omega \subset \mathbb{R}^n, f \in L^2(\Omega)$ %and $g \in H^{1/2}(\partial\Omega).$

\subsubsection{Variational Form}
The variational problem is to find $ u \in H^1_0(\Omega)$ such that 
\[ a(u,v) = (f,v), \quad \text{for all} \quad v \in H^1_0(\Omega),\]
where 
\[ a(u,v) = \int_{\Omega} \nabla u \cdot \nabla v \,dx, \quad \text{and}\quad 
(f,v) =\int_{\Omega} f v \,dx. \]


\subsection{Virtual Element Space}
Suppose $\mathcal{T}_h$ form a finite sequence of partitions of $\Omega$.
$$V_h:= \{v_h \in H^1_0(\Omega):v_h|_E \in V_h^E\  \forall \, E \in \mathcal{T}_h\}$$  

\subsection{Discrete Bilinear Form}
Define $a^E:H^1(E)\times H^1(E)\rightarrow \mathbb{R}$\\
i.e., $a^h(u,v)=(\nabla u,\nabla v)_{0,E}$
Where $(\cdot,\cdot)_{0,E}$ is the $L^2$ inner product.

\subsection{Ritz projection}
Define the operator $\Pi^E: V_h^E \rightarrow \mathcal{P}_E$,
\begin{enumerate}
    \item $(\nabla(\Pi^Ev_h-v_h),\nabla p)_{0,E} = 0 \hspace{5pt} \forall p \in \mathcal{P_E}$
    \item $\overline{\Pi^Ev_h} = \overline{v_h}$
\end{enumerate}

\begin{align*}
(\nabla\Pi^E_h,\nabla p)_{0,E} &= (\nabla v_h,\nabla p)\\
&= \sum_{e\in \delta E}\int_e v_hn_e\cdot\nabla pds 
&= \sum_{e\in \delta E} v_hn_e\cdot pds 
\end{align*}

$$P_0v_h = \cfrac{1}{|E_k|}\int_{E_k}v_h$$

$\Pi^Ev_h = \sum_{i=1}^{N_k}s^im_i$

\subsection{Basis functions}
Introduce $\{\varphi_i\}^N_{i=1} \in V_h$ as the basis of an element space. This can be computed using Lagrangian polynomial between two points.\\
I.e., $\phi_i = \cfrac{(x_i-x)(y_i-y)}{(x_{i+1}-x_i)(y_{i+1}-y_i)}$



\subsection{Stiffness Matrix K}
For each element $E \in \mathcal{T}_h$
$$K^E = \int_E\nabla\varphi_i\cdot\nabla\varphi_j dE$$

\begin{equation}
    K = \sum_{E \in \mathcal{T}_h}K^E
\end{equation}

\subsection{Load Vector F}

$$F^E = \int_Ef\varphi_i dE$$
\begin{equation}
    F = \sum_{E \in \mathcal{T}_h}F^E
\end{equation}


\section{The Ritz projection and local stiffness matrix}
$G = BD $and$ \tilde{G} = \tilde{B}D$

\section{Calculation of the matrices}

$B_E := \begin{bmatrix}
    P_0\varphi_1 & \hdots & P_0 \varphi_{N^\text{dof}}\\
    (\nabla m_2,\nabla\varphi_1)_{0,E} & \hdots & (\nabla m_2,\nabla\varphi_{N^\text{dof}})_{0,E} \\
    \vdots & \ddots & \vdots \\
    (\nabla m_{n_k},\nabla\varphi_1)_{0,E} & \hdots & (\nabla m_{n_k},\nabla\varphi_{N^\text{dof}})_{0,E}
\end{bmatrix}$
\\

\begin{equation}
    P_0v_h := \cfrac{1}{|E|}\int_E v_h, \text{for} k\geq 2
\end{equation}

\begin{equation}
    m_{\alpha_1,\alpha_2} := \left(\cfrac{x - x_{\mathcal{D}}}{h_{\mathcal{D}}}\right)^{\alpha_1}
    \left(\cfrac{y - y_{\mathcal{D}}}{h_{\mathcal{D}}}\right)^{\alpha_2}
\end{equation}

\begin{equation}
    m_{\alpha} = \left(\cfrac{x-x_\mathcal{D}}{h_\mathcal{D}},\cfrac{y-y_\mathcal{D}}{h_\mathcal{D}}\right)
\end{equation}

\begin{equation}
    D = \begin{pmatrix}
        m_1^E(V_1) & m_x^E(V_1) & m_y^E(V_1) \\
        \vdots & \vdots & \vdots \\
        m_1^E(V_n) & m_x^E(V_n) & m_y^E(V_n) \\
    \end{pmatrix}
\end{equation}


\section{Example}
% E1_E2_diagram.tex
\newcommand{\MeshDiagram}[1]{%
    \begin{tikzpicture}[scale=#1]
        % Define Coordinates
        \coordinate (A) at (0.5,0);
        \coordinate (B) at (1.5,0);
        \coordinate (C) at (2,0.5*sqrt(3));
        \coordinate (D) at (1.5,sqrt(3));
        \coordinate (E) at (1.5,1+sqrt(3));
        \coordinate (F) at (0.5,1+sqrt(3));
        \coordinate (G) at (0.5,sqrt(3));
        \coordinate (H) at (0, 0.5*sqrt(3));
        
        % Axis
        \draw[help lines, color=gray!50, dashed] (0,0) grid (3,3);
        \draw[->, ultra thick] (0,0) -- (3,0) node[right] {$x$};
        \draw[->, ultra thick] (0,0) -- (0,3) node[above] {$y$};
        
        % Origin Label
        \node (O) at (0,0) [anchor=north east] {0};
        
        % Y-axis Labels
        \foreach \y in {1,2,3}
            \node (Y\y) at (0,\y) [anchor=east] {\y};
        
        % X-axis Labels
        \foreach \x in {1,2,3}
            \node (X\x) at (\x,0) [anchor=north] {\x};
        
        % Lines and Vertex Labels
        \draw (A) node[anchor=south west] {$V^{E_2}_1$} 
              -- (B) node[anchor=south east] {$V^{E_2}_2$} 
              -- (C) node[anchor=east] {$V^{E_2}_3$} 
              -- (D) node[anchor=north east] {$V^{E_2}_4$} 
              -- (E) node[anchor=north east] {$V^{E_1}_3$} 
              -- (F) node[anchor=north west] {$V^{E_1}_4$} 
              -- (G) node[anchor=north west] {$V^{E_2}_5$} 
              -- (H) node[anchor=west] {$V^{E_2}_6$} 
              -- cycle;
              
        \draw (D) node[anchor=south east] {$V^{E_1}_2$}  
              -- (G) node[anchor=south west] {$V^{E_1}_1$};
        
        % Labels for E1 and E2
        \node[black, rectangle] at (1, {0.5*sqrt(3)}) {\huge $E_2$};
        \node[blue, rectangle] at (1, {0.5 + sqrt(3)}) {\huge $E_1$};
    \end{tikzpicture}
}


$$E_1 = \begin{cases}
    V_1 = (\frac{1}{2},\sqrt{3}) \\
    V_2 = (\frac{3}{2},\sqrt{3}) \\
    V_3 = (\frac{3}{2},1+\sqrt{3}) \\ 
    V_4 = (\frac{1}{2},1+\sqrt{3}) \\
\end{cases}
\quad
E_2 = \begin{cases}
    V_1 = (\frac{1}{2},0) \\ 
    V_2 = (\frac{3}{2},0) \\ 
    V_3 = (2,\frac{\sqrt{3}}{2}) \\
    V_4 = (\frac{3}{2},\sqrt{3}) \\
    V_5 = (\frac{1}{2}, \sqrt{3}) \\ 
    V_6 = (0, \frac{\sqrt{3}}{2}) \\
\end{cases}$$\\

\subsection{E$_1$}
For $E_1$ the square the coordinates of the centriod are, $x_{E_1} = (1, \frac{1}{2}+\sqrt{3}).$ The diameter is, $ h_{E_1} = \sqrt{2}$ and the area is $|E_1| = 1$.\\

\subsubsection{Momonial basis}
$
m_1^1 = \left(\cfrac{x - 1}{\sqrt{2}}\right)^{0}\left(\cfrac{y -\frac{1}{2} - \sqrt{3}}{\sqrt{2}}\right)^{0} = 1\\
m_x^1 = \left(\cfrac{x - 1}{\sqrt{2}}\right)^{1}\left(\cfrac{y -\frac{1}{2} - \sqrt{3}}{\sqrt{2}}\right)^{0} = \cfrac{x - 1}{\sqrt{2}}\\
m_y^1 = \left(\cfrac{x - 1}{\sqrt{2}}\right)^{0}\left(\cfrac{y -\frac{1}{2} - \sqrt{3}}{\sqrt{2}}\right)^{1} = \cfrac{y - \frac{1}{2} - \sqrt{3}}{\sqrt{2}}$\\


\subsubsection{DOF}

\(
\begin{array}{ll}
m_x^1\bigg|_{x=\frac{1}{2}} = \cfrac{\frac{1}{2} - 1}{\sqrt{2}} = -\cfrac{1}{\sqrt{2}} & \qquad m_y^1\bigg|_{y=\sqrt{3}} = \cfrac{\sqrt{3} - \frac{1}{2} - \sqrt{3}}{\sqrt{2}} = -\cfrac{1}{\sqrt{2}} \\
m_x^1\bigg|_{x=\frac{3}{2}} = \cfrac{\frac{3}{2} - 1}{\sqrt{2}} = \cfrac{1}{\sqrt{2}} & \qquad m_y^1\bigg|_{y=\sqrt{3}+1} = \cfrac{1+\sqrt{3} - \frac{1}{2} - \sqrt{3}}{\sqrt{2}} = \cfrac{1}{\sqrt{2}}
\end{array}
\)

\subsubsection{D$_1$ Matrix}
$D_1 = \cfrac{1}{\sqrt{2}}\begin{pmatrix}
    \sqrt{2} & -1 & -1\\
    \sqrt{2} & 1 & -1\\
    \sqrt{2} & 1 & 1\\
    \sqrt{2} & -1 & 1\\
\end{pmatrix}$

\subsubsection{Gradient of m}
$\nabla m_1^1 = (0,0)$\\
$\nabla m_x^1 = (\frac{1}{\sqrt{2}},0)$ \\
$\nabla m_y^1 = (0,\frac{1}{\sqrt{2}})$ 

\subsubsection{Lagrangian $\mathbf{\phi}$ basis}
$\phi_1 = \cfrac{(x_2-x)(y_4-y)}{(x_2-x_1)(y_4-y_1)}$\\
$\phi_1 = \cfrac{(\frac{3}{2} - x)(1+\sqrt{3} + y)}{(\frac{3}{2}-\frac{1}{2})(1+\sqrt{3} - \sqrt{3})}$\\
Note the denominator is 1 and the same is true for the rest so they can be written in the following form.\\
$\phi_1 = (\frac{3}{2} - x)(1 + \sqrt{3} - y)$\\
$\phi_2 = (x - \frac{1}{2})(1 + \sqrt{3} - y)$\\
$\phi_3 = (x - \frac{1}{2} )(y - \sqrt{3})$\\
$\phi_4 = (\frac{3}{2} - x)(y - \sqrt{3})$

\subsubsection{$P_0\varphi_i$}
$
P_0\varphi_0 = \iint_E(\frac{3}{2} - x)(1+\sqrt{3}-y)dxdy \\
\hspace*{22pt}= \int_{\frac{1}{2}}^{\frac{3}{2}}(\frac{3}{2} - x)dx \int_{\sqrt{3}}^{1+\sqrt{3}}(1+\sqrt{3}-y)dy\\
\hspace*{22pt}= \frac{1}{4}[3x-x^2]_{\frac{1}{2}}^{\frac{3}{2}}[2(1+\sqrt{3})x - x^2]_{\sqrt{3}}^{1+\sqrt{3}}\\
\hspace*{22pt}= \frac{1}{4}(9/2-9/4 - 3/2+1/4)(2(1+\sqrt{3})(1+\sqrt{3})\\
\hspace*{22pt}= \frac{1}{4}(1)(1)$



\subsubsection{Gradient of $\phi$}
$\nabla\phi_1 = \left(-(1 + \sqrt{3} - y),  -(\frac{3}{2}-x)\right)$\\
$\nabla\phi_2 = \left(1 + \sqrt{3} - y,-(x - \frac{1}{2})\right)$\\
$\nabla\phi_3 = \left(y - \sqrt{3},x - \frac{1}{2}\right)$\\
$\nabla\phi_4 = \left(-(y-\sqrt{3}),\frac{3}{2}-x\right)$\\

\subsubsection{$(\nabla m_x,\nabla\phi)$}
$(\nabla m_x^1,\nabla\phi_1)_{E_1} = 
\left(\frac{1}{\sqrt{2}},0\right)\cdot\left(-(1 + \sqrt{3} - y),  -(\frac{3}{2}-x)\right) = 
-\frac{1}{\sqrt{2}}(1 + \sqrt{3} - y)|_{y=\sqrt{3}} = 
-\frac{1}{\sqrt{2}}$\\
$(\nabla m_x^1,\nabla\phi_2)_{E_1} = 
\left(\frac{1}{\sqrt{2}},0\right)\cdot\left(1 + \sqrt{3} - y),  -(x-\frac{1}{2})\right) = 
\frac{1}{\sqrt{2}}(1 + \sqrt{3} - y)|_{y=\sqrt{3}} = 
\frac{1}{\sqrt{2}}$\\
$(\nabla m_x^1,\nabla\phi_3)_{E_1} = 
\left(\frac{1}{\sqrt{2}},0\right)\cdot\left(y-\sqrt{3}, x-\frac{1}{2}\right) = 
\frac{1}{\sqrt{2}}(y-\sqrt{3})|_{y=\sqrt{3}+1} = 
-\frac{1}{\sqrt{2}}$\\
$(\nabla m_x^1,\nabla\phi_3)_{E_1} = 
\left(\frac{1}{\sqrt{2}},0\right)\cdot\left(-(y-\sqrt{3}), \frac{3}{2}-x)\right) = 
-\frac{1}{\sqrt{2}}(y-\sqrt{3})|_{y=\sqrt{3}+1} = 
\frac{1}{\sqrt{2}}$\\


\subsubsection{$\mathbf{(\nabla m_y,\nabla\phi)}$}
$(\nabla m_y^1,\nabla\phi_1)_{E_1} = 
\left(0,\frac{1}{\sqrt{2}}\right)\cdot\left(-(1 + \sqrt{3} - y),  -(\frac{3}{2}-x)\right) = 
-\frac{1}{\sqrt{2}}(\frac{3}{2}-x)|_{x=\frac{1}{2}} = 
-\frac{1}{\sqrt{2}}$\\
$(\nabla m_y^1,\nabla\phi_2)_{E_1} = 
\left(0,\frac{1}{\sqrt{2}}\right)\cdot\left(1 + \sqrt{3} - y),  -(x-\frac{1}{2})\right) = 
-\frac{1}{\sqrt{2}}(x - \frac{1}{2})|_{x=\frac{3}{2}} = 
-\frac{1}{\sqrt{2}}$\\
$(\nabla m_y^1,\nabla\phi_3)_{E_1} = 
\left(0,\frac{1}{\sqrt{2}}\right)\cdot\left(y-\sqrt{3}, x-\frac{1}{2}\right) = 
\frac{1}{\sqrt{2}}(x-\frac{1}{2})|_{x=\frac{3}{2}} = 
\frac{1}{\sqrt{2}}$\\
$(\nabla m_y^1,\nabla\phi_3)_{E_1} = 
\left(0,\frac{1}{\sqrt{2}}\right)\cdot\left(-(y-\sqrt{3}), \frac{3}{2}-x)\right) = 
\frac{1}{\sqrt{2}}(\frac{3}{2}-x)|_{x=\frac{1}{2}} = 
\frac{1}{\sqrt{2}}$\\

\subsubsection{B$_1$ Matrix}
$B_1 = \frac{1}{4}\begin{pmatrix}
    1 & 1 & 1 & 1 \\
    -d & 1 & -1 & 1\\
    -1 & -1 & 1 & 1 
\end{pmatrix}$

\subsubsection{G Matrix}
$G_1 = \begin{pmatrix}
    1
\end{pmatrix}
$

\subsection{E$_2$}
For $E_1$ the square the coordinates of the centriod are, $x_{E_2} = (1,1).$ The diameter is, $ h_{E_1} = 2$ and the area is $|E_1| = \frac{3\sqrt{3}}{2}$.\\
\\
For $m_1^2 = \left(\cfrac{x - 1}{2}\right)^{0}\left(\cfrac{y - 1}{2}\right)^{0} = 1$\\
For $m_x^2 = \left(\cfrac{x - 1}{2}\right)^{1}\left(\cfrac{y - 1}{2}\right)^{0} = \cfrac{(x-1)}{2}$\\
For $m_y^2 = \left(\cfrac{x - 1}{2}\right)^{0}\left(\cfrac{y - 1}{2}\right)^{1} = \cfrac{y-1}{2}$\\

\subsubsection{Monimals}
$\begin{array}{ll}
m_x^2|_{x=0}\hspace{1.5pt} = \cfrac{0-1}{2} = -\frac{1}{2} &\qquad m_y^2|_{y=0} \hspace{7pt}= \cfrac{0-1}{2} = -\frac{1}{2} \\
m_x^2|_{x=\frac{1}{2}} = \cfrac{\frac{1}{2}-1}{2} = -\frac{1}{4} &\qquad m_y^2|_{y=\frac{\sqrt{3}}{2}} = \cfrac{\frac{\sqrt{3}}{2}-1}{2} = \frac{\sqrt{3}-2}{4} \\
m_x^2|_{x=\frac{3}{2}} = \cfrac{\frac{3}{2}-1}{2} = \frac{1}{4} &\qquad m_y^2|_{y=\sqrt{3}}\, = \cfrac{\sqrt{3}-1}{2} = \frac{\sqrt{3}-1}{2} \\
m_x^2|_{x=2}\hspace{1.5pt} = \cfrac{2-1}{2} = \frac{1}{2} &
\end{array}$

$D_2 = \cfrac{1}{4}\begin{pmatrix}
    4 & -1 & - 2\\
    4 & 1 & - 2\\
    4 & 2 & \sqrt{3} - 2\\
    4 & 1 & 2\sqrt{3} - 2\\
    4 & -1 & 2\sqrt{2} - 2\\
    4 & -2 & \sqrt{3} - 2\\
\end{pmatrix}$

\subsection{B matrix}
$\nabla m_1^2 = (0,0)$\\
$\nabla m_x^2 = (\frac{1}{2},0)$ \\
$\nabla m_y^2 = (0,\frac{1}{2})$ \\

Hence, \\
$P_0m_1 = \int_E\frac{1}{E}$

\subsection{Mesh example.}
The mesh is defined using 3 sets, Vertices, Elements and Boundary. The first is an array of coordinates for each node. The Vertices element set contains a vector for each element. With the index for each node in the element labeled anti-clockwise. The boundary is a vector containing the indices in an anticlockwise order.



% Include the appendix
\chapter{Appendix}

% Reset section counter and modify numbering for the appendix
\setcounter{section}{0}                 % Reset section counter to 0
\renewcommand{\thesection}{\Alph{section}} % Section numbering as A, B, C, etc.
\renewcommand{\thesubsection}{\Alph{section}\arabic{subsection}} % Subsections as A1, A2, etc.
\renewcommand{\thesubsubsection}{\Alph{section}\arabic{subsection}.\arabic{subsubsection}} % Subsubsections as A1.1, A1.2, etc.


% Appendix A
\section{First Appendix}
This section shows the MATLAB code for the VEM method.

\subsection{The main VEM method}

\begin{lstlisting}[style=MatlabStyle]
function u = vem(mesh_filepath, rhs, boundary_condition)
% VEM computes the virtual element solution of a Poisson problem on a polygonal mesh

% SYNOPSIS: u = vem(mesh_filepath, rhs, boundary_condition)
%
% INPUT: mesh_filepath: A string specifying the path to a mesh file
%        rhs:           A handle to a function specifying the PDE forcing function
%        boundary_condition: A handle to a function specifying the boundary condition of the PDE
% 
% OUTPUT: u: A vector of the the degrees of freedom of the virtual element solution to the PDE

mesh = load(mesh_filepath); % Load the mesh from a .mat file
n_dofs = size(mesh.vertices, 1); n_polys = 3; % Method has 1 degree of freedom per vertex
K = sparse(n_dofs, n_dofs); % Stiffness matrix
F = zeros(n_dofs, 1); % Forcing vector
u = zeros(n_dofs, 1); % Degrees of freedom of the virtual element solution
linear_polynomials = {[0,0], [1,0], [0,1]}; % Impose an ordering on the linear polynomials
mod_wrap = @(x, a) mod(x-1, a) + 1; % Utility function for wrapping around a vector

% Loop through elements to compute local stiffness matrix and forcing vector
for el_id = 1:length(mesh.elements)
    vert_ids = mesh.elements{el_id}; % Global IDs of the vertices of this element
    verts = mesh.vertices(vert_ids, :); % Coordinates of the vertices of this element
    n_sides = length(vert_ids); % Start computing the geometric information
    area_components = verts(:,1) .* verts([2:end,1],2) - verts([2:end,1],1) .* verts(:,2);
    area = 0.5 * abs(sum(area_components));
    centroid = sum((verts + verts([2:end,1],:)) .* repmat(area_components,1,2)) / (6*area);
    diameter = 0; % Compute the diameter by looking at every pair of vertices
    for i = 1:(n_sides-1)
        for j = (i+1):n_sides
            diameter = max(diameter, norm(verts(i, :) - verts(j, :)));
        end
    end

    % Initialize D and B matrices for projection
    D = zeros(n_sides, n_polys); D(:, 1) = 1;
    B = zeros(n_polys, n_sides); B(1, :) = 1/n_sides;
    
    % Further calculations for polynomials, gradients, and projection
    for vertex_id = 1:n_sides
        vert = verts(vertex_id, :); % Current vertex and its neighbors
        prev = verts(mod_wrap(vertex_id - 1, n_sides), :);
        next = verts(mod_wrap(vertex_id + 1, n_sides), :);
        vertex_normal = [next(2) - prev(2), prev(1) - next(1)]; % Edge normal
        for poly_id = 2:n_polys
            poly_degree = linear_polynomials{poly_id};
            monomial_grad = poly_degree / diameter; % Gradient of linear polynomial
            D(vertex_id, poly_id) = dot(vert - centroid, poly_degree) / diameter;
            B(poly_id, vertex_id) = 0.5 * dot(monomial_grad, vertex_normal);
        end
    end

    % Assemble the global stiffness matrix and forcing vector
    projector = (B*D) \ B; % Compute the local Ritz projector to polynomials
    stabilising_term = (eye(n_sides) - D * projector)' * (eye(n_sides) - D * projector);
    G = B*D; G(1, :) = 0;
    local_stiffness = projector' * G * projector + stabilising_term;
    K(vert_ids,vert_ids) = K(vert_ids,vert_ids) + local_stiffness; % Copy local to global
    F(vert_ids) = F(vert_ids) + rhs(centroid) * area / n_sides;
end

% Apply boundary conditions and solve the linear system
boundary_vals = boundary_condition(mesh.vertices(mesh.boundary, :));
internal_dofs = ~ismember(1:n_dofs, mesh.boundary);
F = F - K(:, mesh.boundary) * boundary_vals; % Apply the boundary condition
u(internal_dofs) = K(internal_dofs, internal_dofs) \ F(internal_dofs); % Solve
u(mesh.boundary) = boundary_vals; % Set the boundary values
plot_solution(mesh, u)
end
\end{lstlisting}

% Appendix B
\section{Second Appendix}
This section shows the Julia code used in this report. 

\subsection{Mesh Struct}
This section shows the definition of the Mesh structure.


\begin{jllisting}[style=JuliaStyle]
struct Mesh
    boundary::Vector{Int64}    # Indices of boundary vertices
    elements::Vector{Vector{Int64}}  # Indices of vertices for each element
    vertices::Vector{Tuple{Float64, Float64}}  # Coordinates of vertices
end
\end{jllisting}

\subsection{Main VEM method}
This is the main Virtual Element Method. It implements the theoretical technique discussed in the report.

\begin{jllisting}[style=JuliaStyle]
function vem(mesh::Mesh, rhs::Function, boundary_condition::Function; debug::Bool=false, debug_file_path::String="vem_debug_output.md")::Vector{Float64}
    # Initialize problem dimensions
    n_dofs = length(mesh.vertices)  # Number of degrees of freedom, one per vertex
    n_polys = 3  # Number of polynomials in the VEM space (constant + linear terms)

    # Initialize global stiffness matrix, forcing vector, and solution vector
    K = spzeros(Float64, n_dofs, n_dofs)  # Global stiffness matrix
    F = zeros(n_dofs)  # Global forcing vector
    u = zeros(n_dofs)  # Solution vector (degrees of freedom)

    # Linear polynomials (constant, x, y)
    linear_polynomials = [[0, 0], [1, 0], [0, 1]]  # Polynomial basis functions

    # Debugging: open a debug file if enabled
    if debug
        debug_file = open(debug_file_path, "w")
    end

    # Write the table of contents (TOC) for the debug file if debugging is enabled
    if debug
        write(debug_file, "# Debug Output for VEM Method\n\n")
        write(debug_file, "## Table of Contents\n\n")
        write(debug_file, "- [Number of DOFs](#number-of-dofs)\n")
        write(debug_file, "- [Initial Stiffness Matrix (K)](#initial-stiffness-matrix)\n")
        write(debug_file, "- [Initial Forcing Vector (F)](#initial-forcing-vector)\n")

        # Generate TOC dynamically for each element in the mesh
        for el_id in 1:length(mesh.elements)
            write(debug_file, "- [Element $el_id Details](#element-$el_id-details)\n")
            write(debug_file, "  - [Area, Centroid, Diameter](#element-$el_id-area-centroid-diameter)\n")
            write(debug_file, "  - [D and B Matrices](#element-$el_id-d-and-b-matrices)\n")
            write(debug_file, "  - [K and F Matrices](#element-$el_id-k-and-f-matrices)\n")
        end
        write(debug_file, "- [Boundary Conditions](#boundary-conditions)\n")
        write(debug_file, "- [Final Solution (u)](#final-solution)\n\n")
    end

    # Debugging: initial values of K, F, and u
    if debug
        write(debug_file, "[Back to top](#table-of-contents)\n\n")
        write(debug_file, "## Number of DOFs\nNumber of DOFs: $n_dofs\n\n")
        write(debug_file, "## Initial Stiffness Matrix\nInitial K: $K\n\n")
        write(debug_file, "## Initial Forcing Vector\nInitial F: $F\n\n")
        write(debug_file, "## Initial Degrees of Freedom (u)\nInitial u: $u\n\n")
    end

    # Loop over all elements in the mesh to compute local contributions to K and F
    for el_id in 1:length(mesh.elements)
        vert_ids = mesh.elements[el_id]  # Global vertex IDs of this element
        verts = [mesh.vertices[v] for v in vert_ids]  # Coordinates of the element's vertices
        n_sides = length(vert_ids)  # Number of sides of the polygon (element)

        # Debugging: print element details
        if debug
            write(debug_file, "[Back to top](#table-of-contents)\n\n")
            write(debug_file, "## Element $el_id Details\n")
            write(debug_file, "- Vertex IDs: $vert_ids\n")
            write(debug_file, "- Vertices: $verts\n")
            write(debug_file, "- Number of sides: $n_sides\n\n")
        end

        # Compute geometric properties of the element (area, centroid, and diameter)
        verts_array = hcat([collect(v) for v in verts]...)'  # Convert vertex coordinates to matrix form
        area_components = verts_array[:,1] .* verts_array[[2:end; 1],2] .- verts_array[[2:end; 1],1] .* verts_array[:,2]
        area = 0.5 * abs(sum(area_components))  # Compute the area of the polygon

        # Debugging: print geometric details
        if debug
            write(debug_file, "[Back to top](#table-of-contents)\n\n")
            write(debug_file, "### Element $el_id Area, Centroid, and Diameter\n")
            write(debug_file, "- Area components: $area_components\n")
            write(debug_file, "- Area: $area\n")
        end

        # Compute the centroid of the polygon
        centroid = sum((verts_array .+ verts_array[[2:end; 1],:]) .* repeat(area_components, 1, 2), dims=1) / (6 * area)

        # Compute the diameter of the polygon (largest distance between two vertices)
        diameter = 0.0
        for i in 1:(n_sides-1)
            for j in (i+1):n_sides
                diameter = max(diameter, norm(verts_array[i, :] - verts_array[j, :]))
            end
        end

        # Debugging: print centroid and diameter details
        if debug
            write(debug_file, "- Centroid: $centroid\n")
            write(debug_file, "- Diameter: $diameter\n\n")
        end

        # Initialize D and B matrices for the element
        D = zeros(n_sides, n_polys)  # Matrix D for polynomial projections
        D[:, 1] .= 1  # Constant term in polynomial
        B = zeros(n_polys, n_sides)  # Matrix B for polynomial evaluation
        B[1, :] .= 1 / n_sides  # Constant term for all sides

        # Debugging: print initial D and B matrices
        if debug
            write(debug_file, "[Back to top](#table-of-contents)\n\n")
            write(debug_file, "### Element $el_id D and B Matrices\n")
            write(debug_file, "- D Matrix: $D\n")
            write(debug_file, "- B Matrix: $B\n")
        end

        # Fill the D and B matrices with values for each vertex of the element
        for vertex_id in 1:n_sides
            vert = verts[vertex_id]
            prev = verts[mod(vertex_id - 2, n_sides) + 1]  # Previous vertex
            next = verts[mod(vertex_id, n_sides) + 1]  # Next vertex

            # Coordinates of current, previous, and next vertices
            vert_x, vert_y = vert
            prev_x, prev_y = prev
            next_x, next_y = next

            vertex_normal = [next_y - prev_y, prev_x - next_x]  # Normal vector to the edge
            centroid_vec = vec(centroid)  # Centroid vector

            # Fill D and B for higher-degree polynomials (linear terms)
            for poly_id in 2:n_polys
                poly_degree = linear_polynomials[poly_id]  # Polynomial degrees
                monomial_grad = poly_degree / diameter  # Gradient of the monomial

                D[vertex_id, poly_id] = dot([vert_x, vert_y] .- centroid_vec, poly_degree) / diameter
                B[poly_id, vertex_id] = 0.5 * dot(monomial_grad, vertex_normal)
            end
        end

        # Compute the projector and stabilizing term for the stiffness matrix
        projector = (B * D) \ B  # Projector matrix
        stabilising_term = (I - D * projector)' * (I - D * projector)  # Stabilizing term for the stiffness matrix

        # G matrix for additional stability
        G = B * D
        G[1, :] .= 0  # Set the first row to 0 (to satisfy certain stability conditions)

        # Compute the local stiffness matrix for this element
        local_stiffness = projector' * G * projector + stabilising_term

        # Debugging: print projector, G matrix, and local stiffness matrix
        if debug
            write(debug_file, "[Back to top](#table-of-contents)\n\n")
            write(debug_file, "- Projector: $projector\n")
            write(debug_file, "- Stabilising Term: $stabilising_term\n")
            write(debug_file, "- G Matrix: $G\n")
            write(debug_file, "- Local Stiffness: $local_stiffness\n\n")
        end

        # Assemble the local stiffness matrix into the global matrix K
        for i in 1:length(vert_ids)
            for j in 1:length(vert_ids)
                K[vert_ids[i], vert_ids[j]] += local_stiffness[i, j]
            end
        end

        # Compute the contribution to the global forcing vector F
        rhs_value = rhs(centroid)  # Evaluate the right-hand side function at the centroid
        F[vert_ids] .+= rhs_value * area / n_sides

        # Debugging: print element contribution to K and F
        if debug
            write(debug_file, "[Back to top](#table-of-contents)\n\n")
            write(debug_file, "### Element $el_id K and F Matrices\n")
            write(debug_file, "- RHS value: $rhs_value\n")
            write(debug_file, "- Updated K and F Matrices\n\n")
        end
    end

    # Apply boundary conditions
    boundary_values = boundary_condition(mesh.vertices)
    if debug
        write(debug_file, "[Back to top](#table-of-contents)\n\n")
        write(debug_file, "## Boundary Conditions\n")
        write(debug_file, "- Boundary Values: $boundary_values\n\n")
    end

    for idx in mesh.boundary
        u[idx] = boundary_values[idx]  # Set solution values at boundary
        K[idx, :] .= 0  # Zero out the row in K
        K[idx, idx] = 1.0  # Set the diagonal entry to 1
        F[idx] = boundary_values[idx]  # Set the corresponding entry in F
    end

    # Solve the system K * u = F to find the solution
    u = K \ F

    # Debugging: print the final solution
    if debug
        write(debug_file, "[Back to top](#table-of-contents)\n\n")
        write(debug_file, "## Final Solution\n")
        write(debug_file, "- Final Solution u: $u\n")
    end

    # Close the debug file if it was opened
    if debug
        close(debug_file)
    end

    return u  # Return the computed solution
end
\end{jllisting}

\subsection{Plot Functions}
Here is the plotting functions used. 

\subsubsection{Heatmap Plot}
Unlike Matlab Julia does not have a equivalent function to patches. Which complicated the plotting function. 

\begin{jllisting}[style=JuliaStyle]
function plot_heatmap(mesh::Mesh, solution::Vector{Float64}; ptitle =nothing, axis_labels = nothing, colourscheme=:blues, show_colourbar=true, savepath=nothing)
    # Extract the vertices and solution values
    x_coords = [v[1] for v in mesh.vertices]
    y_coords = [v[2] for v in mesh.vertices]

    # Calculate the axis limits based on the mesh vertices
    x_min, x_max = minimum(x_coords), maximum(x_coords)
    y_min, y_max = minimum(y_coords), maximum(y_coords)

    # Create a plot object with aspect ratio and axis limits, conditionally showing the color bar
    p = plot(legend = false, aspect_ratio = :equal, xlims=(x_min, x_max), ylims=(y_min, y_max), 
             colorbar = show_colourbar ? :right : false)

    # Loop through each element to fill with interpolated color
    for element in mesh.elements
        # Extract the x, y coordinates of the vertices in the element
        x_elem = x_coords[element]
        y_elem = y_coords[element]
        
        # Get the solution values for the vertices of the element
        solution_elem = solution[element]

        # Fill the polygon (triangle/polygon in this case) with interpolated color
        plot!(p, x_elem, y_elem, fill_z = solution_elem, seriestype = :shape, lw = 1, colour=colourscheme)
    end

    # Final plot customization
    if axis_labels !== nothing
        xlabel!(p, axis_labels[1])
        ylabel!(p, axis_labels[2])
    end
    if ptitle !== nothing
        title!(p, ptitle)
    end


    # Save the plot if savepath is provided, otherwise display it
    if savepath !== nothing
        savefig(p, savepath)
    else
        display(p)
    end
end
\end{jllisting}

\subsubsection{Wireframe Plot}
This is the wireframe plot.

\begin{jllisting}[style=JuliaStyle]
function plot_wireframe(mesh::Mesh, heights::Vector{Float64}; ptitle=nothing, azimuth=30, elevation=30, solidcolour=nothing, colourscheme=:viridis, savepath=nothing)
    # Extract vertices coordinates
    x_coords = [v[1] for v in mesh.vertices]
    y_coords = [v[2] for v in mesh.vertices]

    # Normalize heights for color mapping
    min_h = minimum(heights)
    max_h = maximum(heights)
    norm_heights = (heights .- min_h) ./ (max_h - min_h)

    # Initialize 3D plot with camera position
    if ptitle !== nothing
        fig = plot3d(title=ptitle, legend=false, camera=(azimuth, elevation))
    else
        fig = plot3d(legend=false, camera=(azimuth, elevation))
    end
    # Loop through each element and plot the wireframe
    for element in mesh.elements
        # Get the x, y, and z coordinates for the current element
        x_el = [x_coords[i] for i in element]
        y_el = [y_coords[i] for i in element]
        z_el = [heights[i] for i in element]

        # Close the polygon by adding the first vertex at the end
        append!(x_el, x_el[1])
        append!(y_el, y_el[1])
        append!(z_el, z_el[1])

        # Determine color based on provided inputs
        if solidcolour !== nothing
            # Use the solid color if provided
            line_colour = solidcolour
        elseif colourscheme !== nothing
            # Use the color scheme if no solid color is provided
            avg_height = mean(norm_heights[element])  # Average height for the element
            line_colour = cgrad(colourscheme)[round(Int, avg_height * 255) + 1]  # Color from gradient
        else
            error("Please provide either a valid color scheme or a solid color.")
        end

        # Plot the edges of the polygon
        plot3d!(x_el, y_el, z_el, linecolor=line_colour, linewidth=1.5)
    end

    # Save the plot if savepath is provided, otherwise display it
    if savepath !== nothing
        savefig(fig, savepath)
    else
        display(fig)
    end
end
\end{jllisting}



\end{document}
