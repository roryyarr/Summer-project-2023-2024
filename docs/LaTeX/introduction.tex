\documentclass[class=article, crop=false]{standalone}

\usepackage{amsmath}
\usepackage{amssymb}

\begin{document}

\section{Introduction}

This paper discusses the Virtual Element Method (VEM), a numerical scheme for solving Poisson problems. VEM expands the functionality of Finite Element Methods (FEM) by allowing more general meshes, enabling it to handle more complicated geometries.

Inspired by Mimetic Finite Differences (MFD), VEM introduces a finite element-like formulation. This innovation led Beirão et al. to name it the Virtual Element Method, due to its similarities to FEM \cite{beirao2013basic}. Unlike FEM, VEM does not require direct computation of the basis functions, and the stiffness matrices can be constructed without an explicit basis.

This feature gives VEM the ability to handle arbitrarily shaped elements, offering greater flexibility for a wide range of applications. As a result, VEM has become a powerful tool for solving partial differential equations over highly unstructured and general meshes, making it suitable for replicating complex geometries.

\end{document}